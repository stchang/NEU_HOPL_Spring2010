% dynamic.tex

\documentclass[12pt]{article}	% YOUR INPUT FILE MUST CONTAIN THESE
\usepackage{url}
\usepackage{graphicx}
\oddsidemargin  -0.4in
\evensidemargin 0.0in
\textwidth      7in
\headheight     -0.5in
\topmargin      0.0in
\textheight     9.0in
\usepackage{amssymb}
\usepackage{amsmath}
\usepackage{xspace}
\usepackage{mathpartir}
\usepackage{alltt}



\begin{document}							% TWO LINES PLUS THE \end COMMAND AT
															% THE END
\newcommand{\Dynamic}{\texttt{Dynamic}\xspace}
\newcommand{\typecase}{\texttt{typecase}\xspace}
\newcommand{\dynamic}{\texttt{dynamic}\xspace}
\newcommand{\wrong}{\texttt{wrong}\xspace}
\newcommand{\deno}[1]{ \ensuremath{[\![#1]\!]} }
\newcommand{\code}[1]{$\mathtt{#1}$}
\newcommand{\pair}[2]{ \ensuremath{\left\langle #1,#2 \right\rangle} }
\newcommand{\pairtt}[2]{ \ensuremath{\left\langle \mathtt{#1,#2} \right\rangle} }


\title{Type Dynamic - Presentation Notes}
\author{Stephen Chang}
\date{2/26/2010}
\maketitle

\subsection*{Motivation}
\begin{enumerate}
	\item Static typecheckers have many advantages but sometimes dynamic typechecking is unavoidable. Type \Dynamic is about embedding dynamic typechecking within statically typed languages.
	\item What static type to give to \code{eval:string\rightarrow\;??}
	\item What static type to give to \code{print:\;??\rightarrow\;string}
	      Need to know type at runtime to dispatch proper type-specific print fn
	\item What static type to give to \code{read:IO\rightarrow\;??}
	      Can be from disk, another process, another computer over the network/internet
	\item Multi-language programs
\end{enumerate}

\subsection*{Background/History}
\begin{enumerate}
	\item languages with disjoint union (finite): Algol-68, Pascal (tagged variant record)
	\item language infinite disjoint union: Simula-67's subclass structure (INSPECT statement allows program to determine subclass of a value at run time)
	\item languages with dynamic typing in static context: CLU (\code{any} type/force), Cedar/Mesa (REFANY/TYPECASE), Modula-2+, Modula-3 (from Cedar/Mesa) -- these languages wanted to support programming idioms from LISP
	\item formalization of language with \Dynamic: Schaffert and Scheifler (1978) gave formal description and denotational semantics for CLU but did not give a soundness theorem -- also required every value to carry type tag at run time
	\item ML had some proposals for adding \Dynamic but ultimately unpublished: Gordon (1980, personal communication), Mycroft (1983, draft)
	\item languages with mechanisms for handling persistent data: Amber, Modula-2+ (pickling)
\end{enumerate}

\subsection*{Abadi, et al. (1989/1991) - Dynamic Typing in a Statically Typed Language}
\begin{enumerate}
	\item Introduce new datatype \Dynamic whose values pairs of value \texttt{v} and type tag \texttt{T} (so \Dynamic values carry type information at runtime) -- like an infinite disjoint union (sum type) -- allow values of different types to be manipulated uniformly
	\item \code{(dynamic\;e:T)} construct creates \Dynamic values and \typecase construct eliminates
	\item \typecase example:
	\begin{verbatim}
	tostring: Dynamic -> String
	  \dv:Dynamic.
	    typecase dv of
	      (v:String) (string-append '"' v '"')
	      (n:Nat) (natToStr n)
	      (X,Y)(f:X -> Y) "<function>"
	      (X,Y)(p:X x Y)
	        (string-append "<" (tostring (dynamic (fst p):X) ","
	                           (tostring (dynamic (snd p):Y) ">")
	      (d:Dynamic) (string-append "dynamic" (tostring d)
	      else "unknown"
	    end
	\end{verbatim}
	nested \typecase example:
	\begin{verbatim}
	\df:Dynamic.\de:Dynamic.
	  typecase df of
	    (X,Y)(f:X -> Y)
	      typecase de of
	        (e:X) (dynamic (f e):Y)
	        else (dynamic "Error":String)
	      end
	    else (dynamic "Error":String)
	  end
	\end{verbatim}
	\item Contribution: formal description of language with \Dynamic values -- cbv simply-typed lambda calculus + \dynamic and \typecase
	\item typecheck and eval rules for \dynamic:\\
	\inferrule{\Gamma \vdash e:T}{\Gamma\vdash(\dynamic\;e:T):\Dynamic}
	\inferrule{\vdash e \Rightarrow v}{\vdash(\dynamic\;e:T)\Rightarrow(\dynamic\;v:T)}
	\item typecheck and eval rules for \typecase:\\
	\inferrule{\Gamma \vdash e:\Dynamic \\\\ 
	           \forall i,\forall\sigma\in Subst_{\vec{X_i}} \; \Gamma[x_i \leftarrow T_i\sigma]\vdash e_i\sigma:T \\\\
	           \Gamma\vdash e_{else}:T}
	          {\Gamma\vdash(\typecase\;e\;\texttt{of} \\\\
	                        \ldots(\vec{X_i})(x_i:T_i)\;e_i\ldots\\\\
	                        \texttt{else}\; e_{else}\\\\
	                        \texttt{end}):T}\\
	 \inferrule{\vdash\;e\Rightarrow(\dynamic\;w:T)\\\\
	            \forall j<k.match(T,T_j) \textrm{fails}\\\\
	            match(T,T_k) = \sigma\\\\
	            \vdash e_k\sigma[x_k\leftarrow w]\Rightarrow v}
	           {\vdash(\typecase\;e\;\texttt{of}\\\\
	            \ldots(\vec{X_i})(x_i:T_i)\;e_i\ldots\\\\
	            \texttt{else}\; e_{else}\\\\
	            \texttt{end})\Rightarrow v}
	 \inferrule{\vdash\;e\Rightarrow(\dynamic\;w:T)\\\\
	            \forall k.match(T,T_k) \textrm{fails}\\\\
	            \vdash e_{else}\Rightarrow v}
	           {\vdash(\typecase\;e\;\texttt{of}\\\\
	            \ldots(\vec{X_i})(x_i:T_i)\;e_i\ldots\\\\
	            \texttt{else}\; e_{else}\\\\
	            \texttt{end})\Rightarrow v}
	\item Theorem: $\forall e, v, T$, if $\vdash e \Rightarrow v$ and $\vdash e:T$, then $v:T$\\
	      Corollary: $\forall e, v, T$ if $\vdash e \Rightarrow v$ and $\vdash e:T$, then $v \neq$ \wrong (because \wrong is not well-typed)
	\item Authors give denotational semantics -- difficulty is assigning meaning to \Dynamic values -- use ideal model of types and Banach Fixed Point theorem from MacQueen, Plotkin, Sethi (1986)
	\item Typechecking is sound: If $e$ is well typed, then $\deno{e}_\rho \neq \wrong$ (for well-behaved $\rho$)\\
	                             proved via: $\forall\Gamma,e,\rho,T$ ($\rho$ consistent with $\Gamma$ on $e$), if $\Gamma\vdash e:T$ then $\deno{e}_\rho \in \deno{T}$\\
	      Evaluation is sound: If $\vdash e \Rightarrow v$, then $\deno{e} = \deno{v}$
\end{enumerate}

%%%%%%%%%%%%%%%%%%%%%%%%%%%%%%%%%%%%%%%%%%%%%%%%%%%%%%%%%%%%%%%%%%%%%%%%%%%%%%
%% Abadi, et al. (1995) Dynamic Typing in Polymorphic Languages
%%%%%%%%%%%%%%%%%%%%%%%%%%%%%%%%%%%%%%%%%%%%%%%%%%%%%%%%%%%%%%%%%%%%%%%%%%%%%%
\subsection*{Abadi, et al. (1995) - Dynamic Typing in Polymorphic Languages}

%% Explicit Polymorphism
\subsubsection*{Explicit Polymorphism}
\begin{enumerate}
	\item Contribution: add explicit polymorphism to language with \Dynamic -- get type abstractions $(\Lambda)$
	\item example:
	\begin{alltt}
	squarePolyFun = 
	  \(\lambda\)df:Dynamic
	    typecase df of
	      (f:\(\forall\)Z.Z\(\rightarrow\)Int)
	        \(\Lambda\)W.\(\lambda\)x:W.f[W](x)*f[W](x)
	      else \(\Lambda\)W.\(\lambda\)x:W.0
	\end{alltt}
	\item need second-order pattern variables that can be type operators in \typecase, example:
	\begin{alltt}
	dynamicApply = 
	  \(\lambda\)df:Dynamic.\(\lambda\)da:Dynamic.
	    typecase fg of
	      \{F,G\} (f:\(\forall\)Z.F(Z)\(\rightarrow\)G(Z))
	        typecase da of
	          \{W\} (a:F(W))
	            dynamic( f[W](a):G(W) )
	\end{alltt}
	if \code{df = dynamic(\;\Lambda Z.\lambda x:Z\times Z.\pair{snd(x)}{fst(x)}:\ldots\;)} \\
	and \code{da = dynamic(\pair{3}{4}:\ldots)} \\
	then \code{F = \Lambda X.X \times X}, \code{G = \Lambda X.X \times X}, and \code{W = Int} \\
	but if \code{df = dynamic(\;\Lambda Z.\lambda x:Z\rightarrow Z.x:\ldots\;)} \\
	and \code{da = dynamic(\lambda x:Int.x:\ldots)} \\
	then \code{F = \Lambda X.X \rightarrow X}, \code{G = \Lambda X.X\rightarrow X}, and \code{W = Int} \\
	
	\item Formally, \Dynamic values added to System $F_\omega$ (simply-typed lambda calculus + type abstractions + type operators)
	\item Having higher order pattern variables in \typecase allows ambiguous matches - example: pattern \code{F(Int)} matches tag \code{Int} with either \code{F=\Lambda X.X} or \code{F=\Lambda X.Int}.
	\item So patterns must be restricted so that they match any tag uniquely (this property is called definiteness) - restrict language to only second-order polymorphism
	\item Authors go not give typing or evaluation rules for explicit polymorphism
\end{enumerate}

%% Implicit Polymorphism
\subsubsection*{Implicit Polymorphism}
\begin{enumerate}
	\item In a language with implicit polymorphism (like ML), some changes need to be made to matching algorithm
	\item \dynamic construct only takes one parameter -- no type tag -- typechecker infers most general type
	\item in \typecase, matched value must be able to be instantiated differently for different uses.
	example:
	\begin{alltt}
	foo = \(\lambda\)df.
	  typecase df of
	    (f:\(\forall\)A.(A\(\rightarrow\)A)\(\rightarrow\)(A\(\rightarrow\)A)\(\pair{\mathtt{f add1}}{\mathtt{f not}}\)
	  else \(\ldots\)
	\end{alltt}
	\item languages with type inference always infer the most general type, but if there were explicit type tags in \Dynamic values, the programmer could have given a less general type so patterns should always match type tags that are more general than the pattern -- called tag instantiation
	\item first order pattern variables not expressive enough to match some examples so we add second order pattern variables like before that can be instantiated to type operators. Example where second order pattern variables are needed:
	\begin{alltt}
	applyTwice = \(\lambda\)df.\(\lambda\)dxy.
	  typecase df of
	    \{F,F'\} (f:F(P)->F'(P))
	      typecase dxy of
	        \{G,H\} (\(\pairtt{x}{y}\):F(G(Q)) \(\times\) F(H(Q)))
	          \(\pairtt{f x}{f y}\)
	      else ...
	   else ...
	\end{alltt}
	\item tag instantiation and second-order pattern variables cause some problems when used together. Second order pattern variables can depend on universal variables, but tag instantiation requires matching with more general types, so you dont know how many universal variables there will be. 
	Example:
	Tag \code{\forall A.(A\times A)\rightarrow A} matches pattern \code{\{F\}(f:\forall A.F(A)\rightarrow A)} with \code{F = \Lambda X.X\times X} but tag \code{\forall A,B.(A\times B)\rightarrow A} should also match the pattern because it is more general, but \texttt{F} does not depend on \texttt{B} (\code{F = \Lambda X.X \times ??}).
	\item Solution is to capture variables that appear in tag but not in a pattern in a tuple \code{P} and have all pattern variables depend on \code{P}. So pattern \code{\{F\}(f:\forall A.F(A)\rightarrow A)} is actually  \code{\{F\}(f:\forall A.F(A;P)\rightarrow A)}. \code{P} gets instantiated at run time. For the previously mentioned tag \code{\forall A,B.(A\times B)\rightarrow A}, \code{P} gets instantiated to \code{(B)}. Since arity of \code{P} is not known at compile type, a special tuple sort must be introduced. (Show type rules?)
	\item authors give typechecking and evaluation rules for implicit polymorphic language with \Dynamic but do not prove any theorems
	
	
\end{enumerate}

%%%%%%%%%%%%%%%%%%%%%%%%%%%%%%%%%%%%%%%%%%%%%%%%%%%%%%%%%%%%%%%%%%%%%%%%%%%%%%
%% Leroy and Mauny (JFP 1993) Dynamics in ML
%%%%%%%%%%%%%%%%%%%%%%%%%%%%%%%%%%%%%%%%%%%%%%%%%%%%%%%%%%%%%%%%%%%%%%%%%%%%%%
\subsection*{Abadi, et al. (1995) - Dynamic Typing in Polymorphic Languages}
\begin{enumerate}
	\item
\end{enumerate}


\bibliographystyle{acm}
\bibliography{dynamic}


\end{document} % THE INPUT FILE ENDS LIKE THIS