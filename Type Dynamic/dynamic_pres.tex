% dynamic.tex

\documentclass[12pt]{article}	% YOUR INPUT FILE MUST CONTAIN THESE
\usepackage{url}
\usepackage{graphicx}
\oddsidemargin  -0.4in
\evensidemargin 0.0in
\textwidth      7in
\headheight     -0.5in
\topmargin      0.0in
\textheight     9.0in
\usepackage{amssymb}
\usepackage{amsmath}
\usepackage{xspace}
\usepackage{mathpartir}



\begin{document}							% TWO LINES PLUS THE \end COMMAND AT
															% THE END
\newcommand{\Dynamic}{\texttt{Dynamic}\xspace}
\newcommand{\typecase}{\texttt{typecase}\xspace}
\newcommand{\dynamic}{\texttt{dynamic}\xspace}
\newcommand{\wrong}{\texttt{wrong}\xspace}
\newcommand{\deno}[1]{ \ensuremath{[\![#1]\!]} }
\newcommand{\code}[1]{$\mathtt{#1}$}


\title{Type Dynamic - Presentation Notes}
\author{Stephen Chang}
\date{2/26/2010}
\maketitle

\subsection*{Motivation}
\begin{enumerate}
	\item Static typecheckers have many advantages but sometimes dynamic typechecking is unavoidable. Type \Dynamic is about embedding dynamic typechecking within statically typed languages.
	\item What static type to give to \code{eval:string\rightarrow\;??}
	\item What static type to give to \code{pring:??\rightarrow\;string}
	      Need to know type at runtime to dispatch proper type-specific print fn
	\item What static type to give to \code{read:IO\rightarrow\;??}
	      Can be from disk, another process, another computer over the network/internet
	\item Multi-language programs
\end{enumerate}

\subsection*{Background/History}
\begin{enumerate}
	\item languages with disjoint union (finite): Algol-68, Pascal (tagged variant record)
	\item language infinite disjoint union: Simula-67's subclass structure (INSPECT statement allows program to determine subclass of a value at run time)
	\item languages with dynamic typing in static context: CLU (\code{any} type/force), Cedar/Mesa (REFANY/TYPECASE), Modula-2+, Modula-3 (from Cedar/Mesa) -- these languages wanted to support programming idioms from LISP
	\item formalization of language with \Dynamic: Schaffert and Scheifler (1978) gave formal description and denotational semantics for CLU but did not give a soundness theorem -- also required every value to carry type tag at run time
	\item ML had some proposals for adding \Dynamic but ultimately unpublished: Gordon (1980, personal communication), Mycroft (1983, draft)
	\item languages with mechanisms for handling persistent data: Amber, Modula-2+ (pickling)
\end{enumerate}

\subsection*{Abadi, et al. (1989/1991) - Dynamic Typing in a Statically Typed Language}
\begin{enumerate}
	\item Introduce new datatype \Dynamic whose values pairs of value \texttt{v} and type tag \texttt{T} (so \Dynamic values carry type information at runtime) -- like an infinite disjoint union (sum type) -- allow values of different types to be manipulated uniformly
	\item \code{(dynamic\;e:T)} construct creates \Dynamic values and \typecase construct eliminates
	\item \typecase example:
	\begin{verbatim}
	tostring: Dynamic -> String
	  \dv:Dynamic.
	    typecase dv of
	      (v:String) (string-append '"' v '"')
	      (n:Nat) (natToStr n)
	      (X,Y)(f:X -> Y) "<function>"
	      (X,Y)(p:X x Y)
	        (string-append "<" (tostring (dynamic (fst p):X) ","
	                           (tostring (dynamic (snd p):Y) ">")
	      (d:Dynamic) (string-append "dynamic" (tostring d)
	      else "unknown"
	    end
	\end{verbatim}
	nested \typecase example:
	\begin{verbatim}
	\df:Dynamic.\de:Dynamic.
	  typecase df of
	    (X,Y)(f:X -> Y)
	      typecase de of
	        (e:X) (dynamic (f e):Y)
	        else (dynamic "Error":String)
	      end
	    else (dynamic "Error":String)
	  end
	\end{verbatim}
	\item Contribution: formal description of language with \Dynamic values -- cbv simply-typed lambda calculus + \dynamic and \typecase
	\item typecheck and eval rules for \dynamic:\\
	\inferrule{\Gamma \vdash e:T}{\Gamma\vdash(\dynamic\;e:T):\Dynamic}
	\inferrule{\vdash e \Rightarrow v}{\vdash(\dynamic\;e:T)\Rightarrow(\dynamic\;v:T)}
	\item typecheck and eval rules for \typecase:\\
	\inferrule{\Gamma \vdash e:\Dynamic \\\\ 
	           \forall i,\forall\sigma\in Subst_{\vec{X_i}} \; \Gamma[x_i \leftarrow T_i\sigma]\vdash e_i\sigma:T \\\\
	           \Gamma\vdash e_{else}:T}
	          {\Gamma\vdash(\typecase\;e\;\texttt{of} \\\\
	                        \ldots(\vec{X_i})(x_i:T_i)\;e_i\ldots\\\\
	                        \texttt{else}\; e_{else}\\\\
	                        \texttt{end}):T}\\
	 \inferrule{\vdash\;e\Rightarrow(\dynamic\;w:T)\\\\
	            \forall j<k.match(T,T_j) \textrm{fails}\\\\
	            match(T,T_k) = \sigma\\\\
	            \vdash e_k\sigma[x_k\leftarrow w]\Rightarrow v}
	           {\vdash(\typecase\;e\;\texttt{of}\\\\
	            \ldots(\vec{X_i})(x_i:T_i)\;e_i\ldots\\\\
	            \texttt{else}\; e_{else}\\\\
	            \texttt{end})\Rightarrow v}
	 \inferrule{\vdash\;e\Rightarrow(\dynamic\;w:T)\\\\
	            \forall k.match(T,T_k) \textrm{fails}\\\\
	            \vdash e_{else}\Rightarrow v}
	           {\vdash(\typecase\;e\;\texttt{of}\\\\
	            \ldots(\vec{X_i})(x_i:T_i)\;e_i\ldots\\\\
	            \texttt{else}\; e_{else}\\\\
	            \texttt{end})\Rightarrow v}
	\item Theorem: $\forall e, v, T$, if $\vdash e \Rightarrow v$ and $\vdash e:T$, then $v:T$\\
	      Corollary: $\forall e, v, T$ if $\vdash e \Rightarrow v$ and $\vdash e:T$, then $v \neq$ \wrong (because \wrong is not well-typed)
	\item Authors give denotational semantics -- difficulty is assigning meaning to \Dynamic values -- use ideal model of types and Banach Fixed Point theorem from MacQueen, Plotkin, Sethi (1986)
	\item Typechecking is sound: If $e$ is well typed, then $\deno{e}_\rho \neq \wrong$ (for well-behaved $\rho$)\\
	                             proved via: $\forall\Gamma,e,\rho,T$ ($\rho$ consistent with $\Gamma$ on $e$), if $\Gamma\vdash e:T$ then $\deno{e}_\rho \in \deno{T}$\\
	      Evaluation is sound: If $\vdash e \Rightarrow v$, then $\deno{e} = \deno{v}$
\end{enumerate}

\bibliographystyle{acm}
\bibliography{dynamic}


\end{document} % THE INPUT FILE ENDS LIKE THIS