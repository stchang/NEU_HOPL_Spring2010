% delimit_pres.tex

\documentclass[letterpaper]{llncs}
%\documentclass[12pt]{article}	% YOUR INPUT FILE MUST CONTAIN THESE
\usepackage{url}
\usepackage{graphicx}
%\oddsidemargin  -0.4in
%\evensidemargin 0.0in
%\textwidth      7in
%\headheight     -0.5in
%\topmargin      0.0in
%\textheight     9.0in
\usepackage{amssymb}
\usepackage{amsmath}
\usepackage{xspace}
\usepackage{delimit}




\begin{document}							% TWO LINES PLUS THE \end COMMAND AT
															% THE END

\title{Delimited Control - Presentation Notes}
\author{Stephen Chang}
\institute{}
\date{4/13/2010}
\maketitle

%%%%%%%%%%%%%%%%%%%%%%%%%%%%%%%%%%%%%%%%%%%%%%%%%%%%%%%%%%%%%%%%%%%%%%%%%%%%%%
%% call/cc
%%%%%%%%%%%%%%%%%%%%%%%%%%%%%%%%%%%%%%%%%%%%%%%%%%%%%%%%%%%%%%%%%%%%%%%%%%%%%%
\section*{call/cc}

call/cc. Who here has seen it before? Who here is comfortable programming with it? Who here likes programming with it? Luckily for all of you, this presentation is not about call/cc. But it is relevant to my topic so I'm going to start with a review of call/cc.

\begin{itemize}
	\item show semantics
	\item show small examples
\end{itemize}

As I mentioned, the focus of this talk is not call/cc, so in this talk, I'm going to assume that call/cc is accepted as something that's useful to have in a programming language. call/cc is useful because it is a very general control operator that can be used to implement a variety of other control operators. What can you do with it?

\begin{itemize}
	\item show exception handling example
\end{itemize}

For more complicated call/cc examples, we turn to my first paper, by Friedman and Haynes. In this paper Friedman and Haynes give several use cases for call/cc. In this paper, we will see that while call/cc is a useful operator, some effort must be invested in order to get it working properly. In particular, call/cc is sometimes too powerful and must be either extended or constrained in order to work properly with other features of a language.


%%%%%%%%%%%%%%%%%%%%%%%%%%%%%%%%%%%%%%%%%%%%%%%%%%%%%%%%%%%%%%%%%%%%%%%%%%%%%%
%% Friedman, Haynes - Constraining Control (1985 POPL)
%% Haynes, Friedman - Embedding Continuations in Procedural Objects (1987 TOPLAS)
%%%%%%%%%%%%%%%%%%%%%%%%%%%%%%%%%%%%%%%%%%%%%%%%%%%%%%%%%%%%%%%%%%%%%%%%%%%%%%
\section*{Constraining Control}%~\cite{Friedman1985ConstrainingControl}

The paper starts with an implementation of a fluid environment. A fluid environment is used to achieve dynamic binding.

\begin{itemize}
	\item show fluid-let example
\end{itemize}

In a language, with call/cc, this implementation may not work all the time, for example, if a piece of code in the body jumps out of the body and into another fluid-let body, the proper environment will not be used. To fix this problem, we need to extend call/cc.

\begin{itemize}
	\item show call/cc-fluid example
\end{itemize}

What we're doing is saving the fluid environment at the time the continuation was captured, and then installing this environment every time the captured continuation is invoked. We must extend the captured continuation to perform this installation and we achieve this by extending call/cc.

Now let's look at another example

\begin{itemize}
	\item show call/cc-oneshot
\end{itemize}


In this example, we are constraining the continuation captured by call/cc in that we only want to allow the captured continuation to be invoked once. A real world application of this code could be bank software where the captured continuations represent queued up banking transactions. 

From this example and the previous one, we can see a general pattern forming. Essentially, the captured continuation is itself captured in a lambda, where other operations are performed before the continuation is invoked. This pattern is one of the primary contributions of this paper. Friedman and Haynes call this extended continuation a continuation object.

Going back to call/cc-oneshot, there is a problem with this code. The problem is that if there is another continuation that overlaps with this one, then it's still possible for the code in this continuation to be executed more than once. For example, if you have a program E[M] where E = E1[E2[]], and both E and E1 are captured by calls to call/cc, even though they are both ``one-shot'', E1 will still be executed twice.

To fix this situation, the authors implement two mechanisms. First they link related continuations so that each continuation knows about other continuations that potentially overlap with it. Secondly, they devise a communication protocol between continuations and use this message system to disable all continuations that share code with a continuation that is invoked.

\begin{itemize}
	\item continuation linkages
	\item continuation messaging
\end{itemize}

So now things are getting a little complicated and I think this is a good point to wrap up my presentation of the paper. What I wanted to demonstrate was that call/cc is sometimes too powerful so it must be harnessed in order to be useful, but doing so is sometimes unwieldy. And so far we have only implemented single features. A language that has more than one of these features would need to combine all these call/cc extensions. To wrap up, here are some more tools that the authors implement:

\begin{itemize}
	\item continuation linkages
	\item continuation messaging
	\item dynamic domains
\end{itemize}

Ultimately the authors utilize all these tools to implement an unwind-protect mechanism: (unwind-protect <body> <postlude>). unwind-protect guarantees that the code in the postlude will always be executed whenever either the body completes executing, or control leaves the body. It's useful for implementing cleanup, like closing files, or to perform other operations that ensure the system is in a stable state. unwind-protect also has a cousin named dynamic-wind, which accepts an additional prelude parameter. The reason I mention this is that we'll see it again later in the talk.

Last observation, it seems that most non-trivial uses of call/cc require a heap (ie - set!).

takeaways:
\begin{itemize}
	\item call/cc is useful but complicated to use
	\item unwind-protect
	\item call/cc usually requires a heap to use
\end{itemize}




%%%%%%%%%%%%%%%%%%%%%%%%%%%%%%%%%%%%%%%%%%%%%%%%%%%%%%%%%%%%%%%%%%%%%%%%%%%%%%
%% Felleisen - Theory and Practice of First-Class Prompts (1988 POPL)
%%%%%%%%%%%%%%%%%%%%%%%%%%%%%%%%%%%%%%%%%%%%%%%%%%%%%%%%%%%%%%%%%%%%%%%%%%%%%%
\section*{The Theory and Practice of First-Class Prompts}%~\cite{Felleisen1998Prompts}

So far we've seen that call/cc is difficult to use. Around the mid to late 80's, some enterprising researcher named Felleisen thought, hey, maybe we should try to formalize the behavior of this beast. And that brings us to my second paper. After working for a while, Felleisen came up with the $\lambda_v\!\!-\!\!C$ calculus, which is an extension of the call-by-value lambda calculus that includes a call/cc-like operator F. 

\begin{itemize}
	\item introduce calculus - terms, evaluation contexts, $\F$ notions of reductions
\end{itemize}

The semantics of $\F$ is very similar to call/cc. $\F$ captures the current continuation as a function and applies its argument to it. These notions of reduction are doing this, one context frame at a time. The $\F$ operator is slowly getting pushed to the root of the term. The problem is that there is no way to get rid of the $\F$ operator. To address this, Felleisen adds a special computation rule that only applies in the empty context.

\begin{itemize}
	\item show special computation rule
\end{itemize}

So now that we have the special rule, we can use the following relation to reason about programs.

\begin{itemize}
	\item show $\Rightarrow$ relation
\end{itemize}

So I previously mentioned that we want to study call/cc, but now we are using this $\F$ operator instead. So let's compare it to call/cc.

\begin{itemize}
	\item compare $\F$ to call/cc
	\item show call/cc implemented with $\F$
\end{itemize}

When asked why he used F instead of call/cc, the author said it was because it made the calculus nicer (less reduction rules).	


The problem with the calculus is that you can no longer reason locally about terms. For example, $\app{\F}{\lam{k}{0}}$ and 0, when viewed as programs, reduce to the same value. But they do not behave the same way when placed in an arbitrary non-empty context. This is not ideal, so to fix this, Felleisen added a prompt operator \#. The prompt delimits the continuation captured by $\F$.

\begin{itemize}
	\item introduce prompt extension to calculus and reduction rules
\end{itemize}

With the special computation rule, it was as if we had an implicit \# at the root of the program. But now prompts are first class values that can appear anywhere.

Felleisen also derives an abstract machine from the calculus. The machine that is presented is a variation of the CEK machine. Machine has 3 registers: control string, environment, and continuation. The continuation is just the evaluation contexts in a different representation. You can also call this the control stack of the program. Each control stack frame is has a tag so you can determine what kind it is. There is an additional \textbf{ret} continuation frame that indicates that a particular subterm has been reduced to a value.

\begin{itemize}	
	\item introduce abstract machine
\end{itemize}

\begin{itemize}	
	\item show examples - interpreter, unwind-protect
	\item show benefit of not needing heap
\end{itemize}




%%%%%%%%%%%%%%%%%%%%%%%%%%%%%%%%%%%%%%%%%%%%%%%%%%%%%%%%%%%%%%%%%%%%%%%%%%%%%%
%% Sitaram, Felleisen - Control Delimiters and their Hierarchies (1990 HOSC)
%%%%%%%%%%%%%%%%%%%%%%%%%%%%%%%%%%%%%%%%%%%%%%%%%%%%%%%%%%%%%%%%%%%%%%%%%%%%%%
\section*{Control Delimiters and Their Hierarchies}%~\cite{Sitaram1990Hierarchies}

Of course, this is a very naive implementation of unwind-protect. In the implementation of Friedman and Haynes, postlude was guaranteed to be executed whenever you jumped out of body. Here, postlude is guaranteed to be executed because you never allow jumps out of body. This indicates a larger problem, and that is that uses of these delimited control operators may conflict with each other. And that brings me to my third paper. For example, say we have catch and throw operators in our language.

\begin{itemize}	
	\item show catch and throw example
\end{itemize}

Now say that there is a catch around this unwind-protect and there is a throw in the body. Unfortunately, the throw will never reach the catch. Here we would have to extend unwind-protect to relay the throw but the general problem still remains.

And the problem is worse than it seems. Say catch and throw were operators built into the language and say the language also wants to give the programmer access to these general delimited control operators, because they are useful, as we established earlier. Now say the programmer writes a piece of code that looks like (catch ... (\# ... throw ...)).

Well now, the problem becomes a language safety issue because the abstractions of the language are revealed to the programmer.

One solution that seems like it could work is that you match up pairs of $\F$ and \# operators. This wont work so well because sometimes one delimiter is matched with many $\F$ operators.

More natural solution, and the one presented in the paper, is to introduce a hierarchy of control operators where each operator has a level. A delimiter delimits everything at or above its level. So the higher the level, the less access you have to the program context. Now we can give our language level operators a low level, and a user control operator a higher level and they wont conflict.


A few more takeaways from this paper:

\begin{itemize}
	\item the authors show how to implement F using call/cc; looks similar to techniques of Friedmand and Haynes, ie really complicated - authors say not to read it on first pass through the paper
	\item all solutions so far do not need a heap
\end{itemize}

%%%%%%%%%%%%%%%%%%%%%%%%%%%%%%%%%%%%%%%%%%%%%%%%%%%%%%%%%%%%%%%%%%%%%%%%%%%%%%
%% Danvy, Filinski - Abstracting Control (1990 LFP)
%%%%%%%%%%%%%%%%%%%%%%%%%%%%%%%%%%%%%%%%%%%%%%%%%%%%%%%%%%%%%%%%%%%%%%%%%%%%%%
\section*{Abstracting Control}%~\cite{Danvy1990AbstractingControl}

Danvy and Filinski take the denotational approach to studying delimited control operators. Also, they show that they can use CPS transformations to get a hierarchy similar to Dorai and Felleisen, but while Dorai's hierarchy was somewhat ad-hoc, by repeated applying cps transformations, a natural hierarchy falls out.

\begin{itemize}
	\item recap standard CPS
	\item give deno for shift and reset
	\item perform 2nd CPS on shift and reset
	\item compare shift to $\F$
\end{itemize}

Danvy and Filinski also present a nifty application where they implement the CPS using shift and reset. Not only do they get a (metacircular) CPS transform, they get a bunch of cleanup operations as well without having to do a separate pass.



\subsection*{Abstract}




\bibliographystyle{acm}
\bibliography{delimit}


\end{document} % THE INPUT FILE ENDS LIKE THIS